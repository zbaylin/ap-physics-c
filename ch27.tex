\documentclass[12pt,twosided,titlepage]{article}

\usepackage{fancyhdr}
\usepackage{graphicx}
\usepackage{psfrag}
\usepackage{cancel}
\usepackage{fullpage}
\usepackage{palatino} % this is the recommended font
% your other packages go here
\usepackage{amsmath,epsfig,latexsym,amssymb,rotating}
% Define some useful tools for these notes
\newenvironment{thinlist}[1][\labelitemi]{\begin{list}{#1}{\topsep=\smallskipamount \parsep=\smallskipamount \itemsep=\smallskipamount}}{\end{list}}
\renewcommand{\baselinestretch}{2}
% your latex commands go here

\begin{document}
\begin{tabular}{|c c c|}
\hline
%& &\\
& Lecture 2 & Jan. 2-2006 \\
%& &\\
& \textbf{Computational Inference}  &  \\
& {\small STAT 440 / 840, CM 461} &  \\
%& &\\
Lecture: Ali Ghodsi & & Scribes: your name \\
\hline
\end{tabular}

%Your note here

% Add a table. This a table of 3 rows and 4 columns
\begin{tabular}{|c|c|c|c|}
  \hline
  % after \\: \hline or \cline{col1-col2} \cline{col3-col4} ...
  1 & 2 & 3 & 4 \\
  \hline
  5 & 6 & 7 & 8 \\
  12 & 11 & 10 & 9 \\
  \hline

  % Add a figure. Make your figures in eps or ps format
\begin{figure}[h]
\center
\includegraphics[width=0.65\columnwidth,
 keepaspectratio]{file name.ps}
\caption{ } \label{fig1}
\end{figure}

%Add a formula
\begin{eqnarray}
  p(x) = \prod_i~~ f_i(x_i, x_{{\pi}_i} )
\end{eqnarray}

or

\[
  p(x) = \prod_i~~ f_i(x_i, x_{{\pi}_i} )
\]

\end{tabular}


\end{document}